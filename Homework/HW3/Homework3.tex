\documentclass{article}
\usepackage{blindtext}

\usepackage[top=1in, bottom=1in, left=1in, right=1in]{geometry}

\usepackage{amsmath}
\usepackage{graphicx}
\usepackage{commath}
\usepackage{siunitx}

\usepackage[urw-garamond]{mathdesign}
\usepackage[T1]{fontenc}

\usepackage[b]{esvect}

\newcommand{\de}{\mathrm{d}}

\begin{document}
\title{Homework 3}
\author{Xueqi Li}
% \date{Feb 4, 2017}
% \email{xueqi.li@stonybrook.edu}

% \begin{abstract}
% Consider a vector (given with respect to a fixed Cartesian basis). Here $t$ means time.
% \[
% \vv{r}(t) = \sin(\pi t)\hat{x} + \cos(\pi t)\hat{y} - \sqrt{7}\hat{z}
% \]
% \end{abstract}

\maketitle

\begin{enumerate}
    \item A charged particle with mass $\mu$ and positive charge $e$ moves in a uniform electric and magnetic field $\vv{E} = E \hat{x}$ and $\vv{B} = B \hat{x}$. The Lorentz force law tells us that $\vv{F} = e(\vv{E}+\vv{v}\times\vv{B})$.
    \begin{enumerate}
        \item Write the equations of motion in cartesian coordinates.\\

        From $\vv{F} = m a$ we can write:
        \[
        e(\vv{E}+\dot{\vv{r}}\times\vv{B}) = m \ddot{\vv{r}}
        \]
        Now we can plug $\vv{E}$ and $\vv{B}$ and $\vv{r} = x\hat{x} + y\hat{y} + z\hat{z}$:
        \begin{eqnarray*}
        e(E \hat{x}+\frac{\de}{\de t}(x\hat{x} + y\hat{y} + z\hat{z})\times\vv{B}\hat{x}) &=& m \frac{\de^2}{\de t^2} (x\hat{x} + y\hat{y} + z\hat{z}) \\
        e(E \hat{x} + B\dot{z}\hat{y}-B\dot{y} \hat{z} ) &=& \mu (\ddot{x}\hat{x} + \ddot{y}\hat{y} + \ddot{z}\hat{z}) \\
        \end{eqnarray*}
        Thus, we have:
        \begin{eqnarray*}
        eE = \mu \ddot{x} \\
        eB\dot{z} = \mu \ddot{y} \\
        -eB\dot{y} = \mu \ddot{z}
        \end{eqnarray*}
        \item Solve the equations\\

        Solve $x$: 
        \begin{eqnarray*}
        eE &=& \mu \ddot{x} \\
        \frac{e}{\mu}E &=& \frac{\de^2 x}{\de t^2} \\
        \frac{e}{\mu}E \iint \de t^2 &=& x \\
        \frac{e}{\mu}E \int (t + C)\de t &=& x \\
        x &=& \frac{e}{\mu}E (\frac{t^2}{2} + C_1t + C_2) \\
        x &=& \frac{e}{\mu}E \frac{t^2}{2} + \frac{e}{\mu}E C_1t + x_0 \\
        x &=& \frac{e}{\mu}E \frac{t^2}{2} + v_x(0)t + x(0)
        \end{eqnarray*}

        Solve $y$ and $z$:
        \begin{eqnarray*}
        \dot{y} &=& -\frac{\mu}{eB}\ddot{z}\\
        0 &=& -\frac{\mu^2}{eB} \frac{\de}{\de t} \ddot{z} - eB\dot{z} \\
        0 &=& -\frac{\de}{\de t} \ddot{z} - \frac{e^2B^2}{\mu^2}\dot{z} \\
        0 &=& [D^3 + \frac{e^2B^2}{\mu^2} D](z) \\
        0 &=& [D(D+i\sqrt{\frac{e^2B^2}{\mu^2}})(D-i\sqrt{\frac{e^2B^2}{\mu^2}})](z)
        \end{eqnarray*}
        This difference equation have solution:
        \begin{eqnarray*}
        z &=& C_1 + C_2e^{\frac{eB}{\mu}it}+C_3 e^{-\frac{eB}{\mu}it}\\
        &=& C_1 + C_2\cos(\frac{eB}{\mu}t)+ C_3\sin(\frac{eB}{\mu}t)
        \end{eqnarray*}

        Now Solve $y$:
        \begin{eqnarray*}
        \dot{z} &=&\frac{\mu}{eB}\ddot{y}\\
        -eB\dot{y} &=&  \frac{\mu^2}{eB}\dddot{y} \\
        -\frac{e^2B^2}{\mu^2}\dot{y} &=&  \dddot{y}\\
        0 &=& \dddot{y} + \frac{e^2B^2}{\mu^2}\dot{y}
        \end{eqnarray*}
        Notice this is same as $z$. Thus we can have the solution:
        \begin{eqnarray*}
        y &=& C_1 + C_2e^{\frac{eB}{\mu}it}+C_3 e^{-\frac{eB}{\mu}it}\\
        &=& C_1 + C_2\cos(\frac{eB}{\mu}t)+ C_3\sin(\frac{eB}{\mu}t)
        \end{eqnarray*}

        Now let say it starting moving at point $(x_0,y_0,z_0)$ with velocity $(\dot{x}_0,\dot{y}_0,\dot{z}_0)$. Thus at this time, $t = 0$. Thus we have for $z$:
        \begin{eqnarray*}
        z_0 &=& C_1 + C_2 \\
        \dot{z}_0 &=& C_3
        \end{eqnarray*}
        Same as $y$:
        \begin{eqnarray*}
        y_0 &=& C_1 + C_2 \\
        \dot{y}_0 &=& C_3
        \end{eqnarray*}
        Notice the constent term for $z$ and $y$ are not the same.
        \item Sketch the particle's motion if it starts at the origin with $\vv{v}_0 = v_0 \hat{y}$
        \\
        \\
        % \\
        % \\
        % \\
        % \\
        % \\
        % \\
        % \\
        % \\
        % \\
        % \\
        % \\
        % \\


        % \begin{eqnarray*}
        % \dot{y} = -\frac{\mu\ddot{z}}{eB}\\
        % eB\dot{z} = \mu \frac{\de}{\de t} (-\frac{\mu\ddot{z}}{eB}) \\
        % eB\dot{z} = \frac{-\mu^2}{eB} \dddot{z} \\
        % -\frac{e^2B^2}{\mu^2} \frac{\de z}{\de t} = \frac{\de^3 z}{\de t^3}\\
        % -\frac{e^2B^2}{\mu^2} \de z = \frac{\de^3 z}{\de t^2} \\
        % -\frac{e^2B^2}{\mu^2} (z + C_1) = \frac{\de^2 z}{\de t^2} \\
        % -\frac{e^2B^2}{\mu^2} \iint(z + C_1)\de t^2 = z
        % % \frac{e^2B^2}{\mu^2}\dot{z} = \frac{\de ^2 \dot{z}}{\de t ^2} \\
        % % \frac{e^2B^2}{\mu^2}\iint \dot{z} \de t^2 = \dot{z} \\
        % \end{eqnarray*}
    \end{enumerate}
    \item Two people, one with mass $m_1$ and the other with mass $m_2$, stand on a stationary sled with mass $M$ on a frozen lake. Assume that the ice is frictionless.
    \begin{enumerate}
        \item The first person jumps off the sled with a speed $u$ relative to the sled. What is his speed? Use conservation of momentum to find the speed of the sled (which is still carrying the second person).\\

        The momentum is conserve. Thus we have $p_i = p_f = 0$. After the jump, we have
        \begin{eqnarray*}
        p_1 &=& m_1 u \\ 
        p_m &=& (M + m_2) v
        \end{eqnarray*}
        where $p_1 + p_m = p_f$. Thus we can have euqation:
        \begin{eqnarray*}
        (M + m_2) v &=& - m_1 u \\ 
        \abs{v} &=& \frac{m_1 u}{M + m_2}
        \end{eqnarray*}
        Thus we have the speed of sled is $\frac{m_1 u}{M}$, the speed of $m_1$ is $u$.
        \item Now the second person jumps with the same speed $u$ relative to the sled. What is his speed? What is the the speed of the sled? \\

        Before the jump, $v = - \frac{m_1 u}{M + m_2}$. Thus, change the frame to the earth, we have $v_2 = u - \frac{m_1 u}{M + m_2}$.

        In the frame of the sled. $p_i = p_f = 0$. Same as question a), we can have
        \begin{eqnarray*}
        M v_s &=& - m_2 u \\ 
        v_s &=& -\frac{m_2 u}{M }
        \end{eqnarray*}
        In the sled frame. And now again change to earth frame:
        \[
        v = -\frac{m_2 u}{M} - \frac{m_1 u}{M + m_2}
        \]
        \item What is the change in the total kinetic energy of the two people and the sled?\\

        Before the jump, $KE_i = 0$. Thus $\Delta KE = KE_f$. Thus we have:
        \[
        KE = \frac{1}{2}\left[m_1u^2 + m_2 \left(u - \frac{m_1 u}{M + m_2}^2\right)^2 + M\left(\frac{m_2 u}{M} + \frac{m_1 u}{M + m_2}\right)^2\right]
        \]
        \item If we assume that $m_1 + m_2 = m_\text{tot}$ is held constant, we can write $m_1 = am_\text{tot}$ and $m_2 = (1 - a)m_\text{tot}$. What value of a maximizes the speed of the sled? \\

        We want to maximizes following function:
        \[
        v = \frac{(1 - a)m_\text{tot} u}{M} + \frac{am_\text{tot} u}{M + (1 - a)m_\text{tot}}
        \]
        To do so we only need to find $a$ such that $\dot{v} = 0$:
        \begin{eqnarray*}
        \dot{v} &=& \frac{\de}{\de a}\frac{(1 - a)m_\text{tot} u}{M} + \frac{\de}{\de a}\frac{am_\text{tot} u}{M + (1 - a)m_\text{tot}} \\
        \dot{v} &=& \frac{\de}{\de a}\frac{am_\text{tot}u}{M+(1-a)mt} - \frac{um_\text{tot}}{M} \\
        0 &=& \frac{\de}{\de a}\frac{am_\text{tot}u}{M+(1-a)mt} - \frac{um_\text{tot}}{M} \\
        \frac{\de}{\de a}\frac{am_\text{tot}u}{M+(1-a)m_\text{tot}} &=& \frac{um_\text{tot}}{M} \\
        \frac{\de}{\de a}\frac{a}{M+(1-a)m_\text{tot}} &=& \frac{1}{M} \\
        \frac{M+m_\text{tot}}{(-am_\text{tot}+M+m_\text{tot})} &=& \frac{1}{M} \\
        a &=& \frac{\sqrt{M^2+Mm_\text{tot}}+M+m_\text{tot}}{m_\text{tot}}
        \end{eqnarray*}
        Notice this $a$ might larger than 1. However, knowing this give us a max value, $v$ is incress in [0,1], we can take $a = 1$ or close to 1.
    \end{enumerate}
    \item Consider a force $\vv{F} = 2x^2\hat{x}-xy\hat{y}$.
    \begin{enumerate}
        \item How much work would this force do along a path that goes along the x-axis from the origin to the point $\hat{x}$ and then up parallel to the y-axis to the point $\hat{x}+\hat{y}$?\\

        \begin{eqnarray*}
        W &=& \int_\gamma F \cdot \de l \\
        W &=& \int_{(0,0)}^{(1,0)} (2x^2\hat{x}-xy\hat{y}) \cdot \de l +  \int_{(1,0)}^{(1,1)} (2x^2\hat{x}-xy\hat{y}) \cdot \de l \\
        W &=& \int_0^1 (2x^2) \de x +  \int_0^1 (-y) \de y \\
        W &=& \frac{2}{3} - \frac{1}{2} = \frac{1}{6}
        \end{eqnarray*}
        \item How much work would this force do along a path given by $y = x^2$, that is, along the path $x\hat{x} + x^2\hat{y}$, again from the origin to the point $\hat{x}+\hat{y}$? Is the force conservative?\\

        \begin{eqnarray*}
        W &=& \int_\gamma F \cdot \de l \\
        &=& \int_0^1 (2x^2 - xy \frac{\de}{\de x} x^2)\de x \\
        &=& \int_0^1 (2x^2 - xx^2 2x)\de x \\
        &=& \int_0^1 (2x^2 - 2x^4)\de x \\
        &=& \frac{4}{15}
        \end{eqnarray*}
        While two work are not the same, the force is not conservative.
        \item For two constants a and b, how much work would this force do along a path $t^a\hat{x}+t^b\hat{y}$, again from the origin to the point $x\hat{x} + x^2\hat{y}$? \\

        \begin{eqnarray*}
        W &=& \int_\gamma F \cdot \de l \\
        &=& \int_0^1 (2x^2\frac{\de}{\de t}t^a - xy \frac{\de}{\de t}t^b) \de t\\
        &=& \int_0^1 (2ax^2t^{a-1} - bxy t^{b-1} \de t \\
        &=& \int_0^1 (2at^{2a}t^{a-1} - bt^{a+b} t^{b-1} \de t \\
        &=& \int_0^1 2at^{2a+a-1}\de t - \int_0^1 bt^{a+2b-1}\de t \\
        &=& \frac{2}{3} - \frac{b}{a+2b}
        \end{eqnarray*}
    \end{enumerate}
        \item Consider a frictionless table with a hole in the center. A string passes through the hole. On the table, attached to the string, there is mass $m$ moving in a circle with radius $r_0$ and angular velocity $\omega_0$. The string has just enough tension to keep the mass moving around in a circle.
        \begin{enumerate}
            \item What is angular momentum $L_0$ of the mass? What is its kinetic energy $KE_0$?\\

            To find the angular momentum:
            \begin{eqnarray*}
            L_0 &=& I\omega \\
            &=& mr^2 \omega
            \end{eqnarray*}
            To find the kinetic energy:
            \begin{eqnarray*}
            KE &=& \frac{1}{2} m v^2\\
            &=& \frac{1}{2} m \omega^2r^2
            \end{eqnarray*}
            where at this time $r = r_0$, $\omega = \omega_0$.
            \item The string is now gradually shortened. What is the angular momentum $L(r)$ as a function of $r$? What is the angular velocity $\omega(r)$ as a function of $r$? What is the kinetic energy energy $KE(r)$ as a function of r? \\

            Notice the angular momentum conserve:
            \[
            L(r) = L_0 = mr_0^2 \omega_0
            \]
            Thus we have
            \begin{eqnarray*}
            mr_0^2 \omega_0 &=& I(r) \omega(r) \\
            mr_0^2 \omega_0 &=& mr^2 \omega(r) \\
            \omega(r) = \frac{r_0^2 \omega_0}{r^2}
            \end{eqnarray*}
            To find the kinetic energy:
            \begin{eqnarray*}
            KE &=& \frac{1}{2} m v^2\\
            &=& \frac{1}{2} m (\frac{r_0^2 \omega_0}{r^2})^2r^2
            \end{eqnarray*}
            \item What is the tension in the string (the force on the mass) needed to keep the motion circular as a function of $r$? What is the work done by this force as the radius changes from $r_0$ to $r$? How does this compare to the change in the kinetic energy?\\

            The force is point to the center and given as:
            \[
            F = ma = m \frac{v^2}{r} = m \frac{r_0^4 \omega_0^2}{r^3}
            \]
            Now we can calculate the work:
            \begin{eqnarray*}
            W &=& \int_{r_0}^r F \cdot \de r \\
            W &=& \int_{r_0}^r m \frac{r_0^4 \omega_0^2}{r^3} \de r \\
            W &=& mr_0^4\omega_0^2 \int_{r_0}^r \frac{1}{r^3} \de r \\
            W &=& \frac{1}{2}mr_0^4\omega_0^2 (\frac{1}{r^2} - \frac{1}{r_0^2})
            \end{eqnarray*}
            And we can compute the kinetic energy:
            \[
            \Delta KE = \frac{1}{2} m (\frac{r_0^2 \omega_0}{r^2})^2r^2 - \frac{1}{2} m \omega_0^2r_0^2 = \frac{1}{2}m \omega_0^2r_0^2(\frac{r_0^2}{r^2} - 1) = W
            \]
        \end{enumerate}
        \item Calculate the center of mass for a cylindrically symmetric bowl made of clay with uniform density $\rho$ and with a shape described by an inner and an outer curve. The outer curve goes from $z \in [0,1]$ and is given by
        \[
        R_\text{out} = \frac{1}{2} + \sqrt{z}
        \]
        The inside of the bowl goes from $z \in [\frac{1}{8}.1]$ and is given by
        \[
        R_\text{in} = \frac{3}{8} + \sqrt{z-\frac{1}{8}}
        \]\\

        First find the mass:
        \begin{eqnarray*}
        M &=& \rho V \\
        &=& \rho [\int_0^1 (\frac{1}{2} + \sqrt{z})^2\pi \de z - \int_0^1 (\frac{3}{8} + \sqrt{z-\frac{1}{8}})^2\pi \de z]\\
        &=& \rho \pi [\int_0^1 (\frac{1}{2} + \sqrt{z})^2 \de z - \int_0^1 (\frac{3}{8} + \sqrt{z-\frac{1}{8}})^2 \de z]
        \end{eqnarray*}
        Now we have:
        \begin{eqnarray*}
        z_{cm} = \frac{1}{M} [\int_0^1 (\frac{1}{2} + \sqrt{z})^2\pi z \de z - \int_0^1 (\frac{3}{8} + \sqrt{z-\frac{1}{8}})^2\pi z \de z]
        \end{eqnarray*}
    \end{enumerate}




% \begin{eqnarray*}
\end{document}